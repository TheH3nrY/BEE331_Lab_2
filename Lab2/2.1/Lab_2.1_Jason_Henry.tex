\documentclass[9pt,a4paper]{report}
\usepackage{mwe}
\usepackage{listings}
\usepackage{amsmath}
\usepackage{graphicx}
\usepackage{subfig}
\usepackage{float}
\usepackage{xcolor}
\usepackage{multirow}
\usepackage{hyperref}
\usepackage{fancyhdr}
\usepackage{sectsty}
\usepackage[dvipsnames]{xcolor}
\usepackage{soul}
\usepackage[compact]{titlesec}
\usepackage{float}
\usepackage[left=0.5cm,right=0.5cm,top=0.5cm,bottom=0.5cm]{geometry}
\graphicspath{{Spice}}

\newcommand*{\nchapter}[1]{%
	\chapter*{#1}%
	\addcontentsline{toc}{chapter}{#1}
	\vspace{-14mm}}
\newcommand*{\nsection}[1]{%
	\section*{#1}%
	\addcontentsline{toc}{section}{#1}}
\newcommand*{\nsubsection}[1]{%
	\subsection*{#1}%
	\addcontentsline{toc}{subsection}{#1}}
\newcommand*{\nsubsubsection}[1]{%
	\subsubsection*{#1}%
	\addcontentsline{toc}{subsubsection}{#1}}

\chaptertitlefont{\large}
\sectionfont{\normalsize}
\fontsize{9}{11}\selectfont
\begin{document}
	\begin{titlepage}
		\centering
		\vspace*{1.5in}
		\includegraphics[width=0.15\textwidth]{W-Logo_Purple_RGB}\par\vspace{1cm}
		{\LARGE \textsc{University of Washington}\par}
		\vspace{1cm}
		{\Large \textsc{BEE331 Lab 2.1}\par}
		\vspace{1.5cm}
		{\huge\bfseries \par}
		\vspace{2cm}
		{\Large\itshape 2301991\hspace{55pt}2130474\par}
		{\Large\itshape Jason Truong\hspace{31pt}Henry Haight\par}
		\vfill
		supervised by\par
		Prof.~Joseph \textsc{Decuir}
		\date{2024\\ January}
		\vfill
		% Bottom of the page
		{\large \today\par}
		\vspace*{1.5in}
	\end{titlepage}
	
	\nchapter{Characterising MOSFET; I-V Curve}
	
	\nsection{Design Objective}
	In this lab, we introduce ourselves to the MOSFET, we characterise its function $V_{DS}$ and $I_D$ curve; related to Triode, Threshold, and Saturation.
	\begingroup
	\renewcommand{\cleardoublepage}{}
	\renewcommand{\clearpage}{}
	\nsection{Circuit Design Outline}
	\endgroup
	With an NMOS (2N7000) Transistor in series from Voltage Input $V_{DD}=12V$ to a resistor ($R_D=100\Omega$), the voltage drop in $V_{DS}=V_D-V_S$, to ground. In the Gate $V_G$ over the capacitor of the MOSFET; in series with a resistor ($R_G=1k\Omega$), being supplied with it's own voltage source; $V_{GG}$.\\\\
	The demonstrated circuits below use a sinusoidal sweep with $V_{DD}$ and $V_{GG}$ respectively to demonstrate the $I_D$ relation over a voltage-change.
	\begin{figure}[hp!]
		\centering
		\caption{\centering 2N7000 NMOS}
		\subfloat[\centering LTSpice + Rudimentary Schematic $I_D$ vs $V_{GS}$ Sweep]{\includegraphics[width=9cm]{Spice Sim Sweep}}\hfil
		\subfloat[\centering LTSpice + Rudimentary Schematic $I_D$ vs $V_{DS}$ Sweep]{\includegraphics[width=9cm]{ID vs VGS Sweep}}
	\end{figure}
	\begin{figure}[hp!]
		\ContinuedFloat
		\centering
		\subfloat[\centering 2N7000 NMOS Transistor MOSFET Circuit]{\includegraphics[width=18cm]{2N7000 Transistor Loaded}}
	\end{figure}
	
	\newpage
	\nsection{Descriptions of Measurements \& Calculations}
	\nsubsection{Analysis}
	\begin{itemize}
		\item \textbf{i. Use your plot (or data) to find the value of $V_{GS}$ which just starts to produce a non-zero drain current. This is the threshold voltage $V_t$ of the MOSFET under test.}
		\subitem See below for tables and graphs of [$I_D$ vs $V_{GS}$], and [$I_D$ vs $V_{GS}$].
		\item \textbf{ii. How close are the measured and calculated values of $V_{DS}$ at the boundary of triode and saturation regions of the MOSFET?}
		\subitem \textit{$V_{GS}$ vs $I_D$}: From when the current has a measurable rating ($1.8V$), saturates at a $0.6V$ difference ($2.4V$).
 		\subitem \textit{$V_{DS}$ vs $I_D$}: From when the current plateaus at its greatest amount ($1V$ to $1.5V$), saturates at a $.5V$ difference (@$1.5V$). Then proceeds to break the component's specifications by breaching current at around $5V$
		\item \textbf{What model parameters of the MOSFET would you adjust (and how) to match the experimental results?}
		\subitem To simulate the $I_D$ change over a voltage change for [Saturation] and [To $10V$] respectively, we AC-swept over a region.\\
		Aside from the simulation-method, here is the Github link to the .lib model we used to simulate the \href{https://github.com/pepaslabs/LTSpice-parts/blob/master/parts/transistor/mosfet/nchannel/2N7000.nxp.lib}{2N7000 component}.
		\item \textbf{Do the values of $k_n$ obtained from measurements agree with $k_n$ obtained from the model?}
		\subitem Simply, yes; given $k_n=(\frac{I_D*2}{V_{RD}-1.6})^2$
	\end{itemize}
	\begin{figure}[hp!] %140, 
		\centering
		\caption{NMOS Circuits}
		\subfloat[$V_{GS}$ vs $I_D$ Graph]{\includegraphics[width=9cm]{V_GS vs I_D Graph}}
		\subfloat[Lab 2 2-2A-I Table]{\includegraphics[width=9cm]{Lab 2 2-2A-I Table}}
	\end{figure}
	\begin{figure}[hp!]
		\ContinuedFloat
		\centering
		\subfloat[$I_D$ vs $V_DS$]{\includegraphics[width=9cm]{I_D vs V_DS}}
		\subfloat[Lab 2 2-2A-I Table]{\includegraphics[width=9cm]{Lab 2-2-2A-II Table}}
	\end{figure}
	
	\newpage
	\nchapter{Addendum Pages}
	\begin{figure}[hp!]
		\centering
		\caption{Jason Truong Addendum}
		\subfloat[\centering Lab Design   Calculations]{\includegraphics[width=17cm]{Theoretical Calculation of Ideal Diode}}
	\end{figure}
	\newpage
	\nsection{Bibliography}
	\textbf{Cited:}\\
	\begin{itemize}
		\item Lab 1 Manual
		\item Sedra, Adel, and Kenneth Smith. Microelectronic Circuits. S.L., Oxford Univ Press Us, 2019.
	\end{itemize}
	\begin{figure}[!h]
		\subfloat[Keep believing in yourself.]{\includegraphics[width=\linewidth]{IMG_20240705_171509323}}
	\end{figure}
\end{document}